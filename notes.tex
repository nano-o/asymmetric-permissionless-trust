\documentclass[11pt,letterpaper]{article}

% \usepackage[lmargin=0in,rmargin=2.5in]{geometry}
\setlength{\marginparwidth}{1.5in}
\usepackage{amsmath,amssymb,amsfonts}
\usepackage{xcolor}
\usepackage{amsthm}
\usepackage{mathtools}
\usepackage[]{hyperref}
\hypersetup{
  colorlinks=false,
}
\usepackage{cleveref}
% \usepackage[disable]{todonotes}
\usepackage{todonotes}
\newcommand{\mytodo}[1]{\todo[size=\scriptsize]{#1}}

\usepackage{biblatex}
\addbibresource{library.bib}

\newtheorem{property}{Property}
\newtheorem{definition}{Definition}
\newtheorem{claim}{Claim}
\newtheorem{rrule}{Rule}
\newtheorem{thm}{Theorem}

\newcommand{\note}[1]{{\color{red}#1}}

\title{Notes on Asymmetric Distributed Trust in a Permissionless System}
\date{\today}
\author{Giuliano Losa\\giuliano@losa.fr}

\begin{document}

\maketitle

\section{Federated Fail-Prone Systems}
\label{sec:basic}

We consider a set of processes $\mathcal{P}=\left\{p_1,...,p_n\right\}$ in a message-passing system with Byzantine failures.
We assume that processes do not know $\mathcal{P}$.
Instead, each process $p_i$ trusts a subset $T_i$ of the processes.
However, $p_i$ trusts members of $T_i$ only to the extent that they do not fail, and $p_i$ makes assumptions about those failures using a fail-prone system $\mathcal{F}_i$ over $T_i$ (i.e.\ $\mathcal{F}_i\subseteq 2^{T_i}$);
$\mathcal{F}_i$ represents the subsets of $T_i$ that $p_i$ thinks can fail together in an execution.
Note that, in a permissionless system, we expect that two different processes $p_i$ and $p_j$ might have non-intersecting trusted sets, i.e.\ it is well possible that $T_i\cap T_j=\emptyset$.

We described the trusted sets and fail-prone systems of the processes using a federated fail-prone system, which is an array $\mathbb{F}=\left[\left(T_1,\mathcal{F}_1\right),...,\left(T_n,\mathcal{F}_n\right)\right]$ that associates to each process $p_i$ a set of trusts processes $T_i$ and a fail-prone system $\mathcal{F}_i$ over $T_i$.
We now consider a fixed federated fail-prone system $\mathbb{F}$.

We now make a few definitions relating the processes' failure assumptions and the actual failures happening in an execution.
We say that a set $A$ is fail-compatible if and only if, for every process $p_i$ in $\mathcal{P}\setminus A$, there exists $F\in\mathcal{F}_i$ such that $T_i\cap A \subseteq F$.
Given an execution, we say that a process that does not fail during an execution is well-behaved in that execution, and otherwise it is faulty.
We say that an execution is fail-compatible if and only if the set of faulty processes in that execution, noted $A$, is fail-compatible.

Given a federated fail-prone system, we would like to obtain a quorum system that enables solving problems such as reliable broadcast and consensus in fail-compatible executions.
Since processes each make their own failure assumptions, we consider quorum systems in which each process has its own set of quorums, as formalized in federated quorum systems.
A federated quorum system $\mathbb{Q}$ is an array $\left[\mathcal{Q}_1,...,\mathcal{Q}_n\right]$ where, for every process $p_i$, $\mathcal{Q}_i$ is a quorum system over $\mathcal{P}$ (i.e.\ $Q_i\subseteq 2^{\mathcal{P}}$).
We would like a quorum system to satisfy the following two properties.
\begin{itemize}
  \item[Availability] In every fail-compatible execution, every well-behaved process $p_i$  has at least one quorum that is exclusively well-behaved.
  \item[Consistency] In every fail-compatible execution, for every two well-behaved processes $p_i$ and $p_j$, for every two quorums $Q_i$ of $p_i$ and $Q_j$ of $p_j$, $Q_i$ and $Q_j$ have a well-behaved member in common.
\end{itemize}

Note that, since our goal is for our construction to be useful in permissionless systems, we do not want to require that processes' trusted sets intersect, or some other pairwise condition on processes' fail-prone systems, as that would mean that a process that joins the system has to synchronize with every other process already in the system.
Thus, the simple approach consisting of defining the quorums of a process $p_i$ as the sets of the form $T_i\setminus F$, where $F\in\mathcal{F}_i$, does not make sense in this context.

\subsection{The Canonical Quorum System}

We now define the canonical quorum system $\overline{\mathbb{F}}$ for $\mathbb{F}$.
However, let us first make an important observation.
If we are to define the quorums of a process $p_i$ in terms of the trusted sets and fail-prone sets of other processes in the system, we face the challenge that faulty processes may lie about their trusted sets and fail-prone sets.
Thus, because they depend on the possible lies of faulty processes, the quorums of a well-behaved process must depend on the execution we are considering.
Thus, the Canonical Quorum System maps each execution to a quorum system that soundly approximates the worst that faulty nodes can do.

Before we proceed, let us define the notion of slice.
A set $S$ is a slice of process $p_i$ if and only if there is a fail-prone set $F\in \mathcal{F}_i$ such that $S=P_i\setminus F$.
(Note that, in a fail-compatible execution, every process that does not fail can count on one of its slices to be well-behaved).

Since faulty processes may lie about their slices, we will also use the notion of observed slice, which is a slice as observed by a process in a given execution.
If $p_i$ is a well-behaved process, then the slices of $p_i$ as observed by process $p_i$ are exactly $p_i$'s slices.
Otherwise, if $p_j$ is faulty, then the slices of $p_i$ as observed by $p_i$ may be arbitrary.

We now define the notion of effective quorum, which is a set of processes deemed a quorum by a process in an execution, and the corresponding Effective Quorum System.
\begin{definition}[Effective Quorum System]
  The Effective Quorum System is such that a set $Q$ is an effective quorum of $p_i$ if and only if:
  \begin{enumerate}
    \item For every $p_j$ in $Q$, there is an observed slice $S$ of $p_j$, as observed by $p_i$, such that $S \subseteq Q$.
    \item There is a fail-prone set $F$ of $p_i$ such that $T_i\setminus F \subseteq Q$.
  \end{enumerate}
\end{definition}

Note that the effective quorums of a process $p_i$ not only depend on which processes are faulty in an execution, but also on what slices $p_i$ observes.
In order to simplify our analysis, we now define the Canonical Quorum System for $\mathbb{F}$, where the quorums of a process only depend on what processes are faulty in an execution.
Crucially, the Canonical Quorum System is a sound abstraction of effective quorums.

\begin{definition}[Canonical Quorum System]
  Given an execution $e$ with faulty set $A$, the Canonical Quorum System $\overline{\mathbb{F}}(e)$ is such that, for every $p_i$, for every set $Q$ of processes, $Q$ is a quorum of $p_i$ if and only if:
\begin{enumerate}
  \item For every $p_j$ in $Q\setminus A$, there is a slice $S$ of $p_j$ such that $S \subseteq Q$.
    \label{def:quorum_1}
  \item There is a fail-prone set $F$ of $p_i$ such that $T_i\setminus F \subseteq Q$.
\end{enumerate}
\end{definition}

Note that, in~\Cref{def:quorum_1} above, we exclude faulty processes from the range of the outermost quantifier.
Thus, regardless of what slices $p_i$ observes, every effective quorum of $p_i$ is a quorum of $p_i$ in the Canonical Quorum System.
This is why the approximation is safe.

\subsubsection{Availability}

From the definition of fail-compatible execution, we immediately get that, for every fail-compatible execution $e$ and for every non-faulty process $p_i$ in that execution, $p_i$ has at least one non-faulty quorum: this quorum is the set $\mathcal{P}\setminus A$ (where $A$ is the set of faulty processes in $e$).
Thus, $\mathcal{Q}$ satisfies the Availability Property.

Moreover, because the observed slices of well-behaved nodes are exactly the slices of well-behaved nodes, $\mathcal{P}\setminus A$ is also an effective quorum of every process.
Thus, the Effective Quorum System also satisfies the Availability Property.

\subsubsection{Consistency}

Let us now turn our attention to the Consistency Property.
In General, there is no reason for Consistency to hold.
Instead, we now provide a condition on the fail-prone system $\mathbb{F}$ that imply that, for every fail-compatible execution $e$, the Canonical Quorum System $\mathcal{Q}(e)$ satisfies the Consistency Property.

We start with a few preliminary definitions.
Define a slice choice $\mathcal{S}$ as an array $\left[S_1,...,S_n\right]$ of slices, where, for every $i$, $S_i$ is a slice of $p_i$.
Note that a slice-choice array can be seen as a directed graph, called slice-choice graph, whose nodes are the processes and where there is an edge between $p_i$ and $p_j$ if and only if $p_j$ is in the slice $S_i$.
Finally, if $A$ is a subset of the vertices of $G$, we write $G-A$ for the graph obtained by removing from $G$ all edges $(p,q)$ such that either $p\in A$ or $q\in A$.

\begin{definition}[Property $B^3(\mathbb{F})$]
  Property $B^3(\mathbb{F})$ holds if and only if for every two processes $p_i$ and $p_j$, for every two slice-choice graphs $G_i$ and $G_j$, for every fail-compatible set $A$ where $p_i\not\in A$ and $p_j\not\in A$, the set of vertices reachable from $p_i$ in $G_i-A$ intersects the set of vertices reachable from $p_j$ in $G_j-A$.
\end{definition}

Intuitively, Property $B^3(\mathbb{F})$ says that, in every fail-compatible execution, two processes can always transitively hear from a third common process despite failures.

\begin{claim}
  \label{thm:canonical}
  If $B^3(\mathcal{F})$ holds, then the Canonical Quorum System for $\mathbb{F}$ satisfies the  Consistency property.
\end{claim}
\begin{proof}
  Consider a fail-compatible execution and let $A$ be the set of processes that fail during the execution.

  Let us now show that $\overline{\mathbb{F}}$ satisfies the Consistency property.
  Consider two well-behaved processes $p_i$ and $p_j$, a quorum $Q_i$ of $p_i$, and a quorum $Q_j$ of $p_j$.
  We must show that $Q_i$ and $Q_j$ have a well-behaved member in common (i.e.\ a common member in $\mathcal{P}\setminus A$).

  By definition of $\overline{\mathbb{F}}$, there exists a slice-choice array $\mathcal{S}_Q=\left[S_Q^1,...,S_Q^n\right]$ such that, for every $p_k\in (Q\cup\{p_i\})\setminus A$, $S_Q^k\subseteq Q$.
  Similarly, there exists a slice-choice array $\mathcal{S}_{Q'}=\left[S_{Q'}^1,...,S_{Q'}^n\right]$ such that, for every $p_k\in (Q'\cup\{p_j\})\setminus A$, $S_{Q'}^k\subseteq Q'$.

  Now consider the slice-choice graphs $G_Q$ and $G_{Q'}$ corresponding to $\mathcal{S}_Q$ and $\mathcal{S}_{Q'}$, respectively.
  By Property~$B^3(\mathcal{F})$, we obtain that the set of vertices reachable from $p_i$ in $G_Q-A$ intersects the set of vertices reachable from $p_j$ in $G_{Q'}-A$.
  Moreover, observe that the set of vertices reachable from $p_i$ in $G_Q-A$ is a subset of $Q$.
  Similarly, observe that the set of vertices reachable from $p_j$ in $G_{Q'}-A$ is a subset of $Q'$.
  Thus, we conclude that $Q$ and $Q'$ have a well-behaved member in common.
\end{proof}

\end{document}

\maketitle


