\documentclass[11pt]{article}

\usepackage[margin=1in]{geometry}
\usepackage{amsmath,amssymb,amsfonts}
\usepackage{xcolor}
\usepackage{amsthm}
\usepackage{mathtools}
\usepackage[]{hyperref}
\hypersetup{
  colorlinks=false,
}
\usepackage{cleveref}
\usepackage{todonotes}

\newtheorem{property}{Property}
\newtheorem{definition}{Definition}
\newtheorem{claim}{Claim}
\newtheorem{rrule}{Rule}
\newtheorem{thm}{Theorem}

% \title{Consensus with Permissionless Quorums in the Stellar Network}
% A reference to Cachin's paper:
\title{Asymmetric Distributed Trust in a Permissionless System}
\date{\today}
\author{Giuliano Losa}

\begin{document}

\maketitle

\abstract{
  Fail-prone systems capture assumptions about which processes may fail together during an execution of a distributed system, and to every fail-prone system corresponds a canonical quorum system that can be used to solve tasks such as read-write registers and consensus. A disadvantage of quorum systems is that require prior system-wide agreement on the notion of quorum, which is hard to achieve in a permissionless system. Even asymmetric quorums systems, where each process makes its own assumptions, require that every two processes make compatible assumptions, which again is hard to achieve in a permissionless system. The Stellar Network offers a practical scheme to obtain a federated byzantine quorum system (FBQS for short), which can be used to solve consensus, in a permissionless network. However, the relationship between the FBQS model and traditional quorum systems is not well-understood. In this paper we show that, similarly to traditional quorum systems, a FBQS corresponds to assumptions about failures that are formalized using a fail-prone system.
}

\section{Introduction}

% Following Cachin and Tackmann, we define an asymmetric fail-prone system $\mathbb{F}=\left[\mathcal{F}_1,...,\mathcal{F}_n\right]$ as a family of fail-prone systems, where the fail-prone system $\mathcal{F}_i$ is a set of sets of processes that denote the trust assumptions of process $p_i$. However, we interpret the assumption encoded by a fail-prone system differently from Cachin and Tackmann.

A common problem in a distributed system is to solve a task (e.g.\ consensus) given some assumption about the possible failures that may occur. For example, one may describe possible failures using fail-prone systems. A fail-prone system $\mathcal{F}=\left\{F_1,...,F_n\right\}$ is a set of sets of processes called fail-prone sets. $\mathcal{F}$ denotes the assumption that, in a given execution, the set of processes that actually fail and collude to disrupt the system is exactly one of the $F_i$s. However, the processes do not know which one it is.

A quorum system for $\mathcal{F}$ is a set of quorums, which are sets of processes, such that (a) the intersection of two quorums is not a subset of a fail-prone set and (b) for every fail-prone set $F$, there is a quorum disjoint from $F$. This guarantees that, in a given execution where the real set of processes that fail is $F\in\mathcal{F}$, every two quorums have a non-faulty member in common, and there is at least one quorum that is exclusively non-faulty. Given a quorum system, there are well-known algorithms for implementing various task, such as read-write registers or, under eventual synchrony, consensus.

Hirt and Maurer show that there exists a quorum system for $\mathcal{F}$ if and only if no three fail-prone sets cover the whole system (this property is called property $Q^3(\mathcal{F})$). Moreover, every fail-prone system that satisfying $Q^3(\mathcal{F})$ determines a canonical quorum system $\overline{\mathcal{F}}$ where $Q$ is a quorum if and only if $Q$ is the complement of a fail-prone set.%$Q\in\overline{\mathcal{F}}$ if and only if there exists a fail-prone set $F\in\mathcal{F}$ such that $Q=\mathcal{P}\setminus F$.

Cachin and Tackmann generalize fail-prone systems to asymmetric fail-prone systems, where each process makes its own assumptions about which processes may fail together. An asymmetric fail-prone system is a family of fail-prone systems $\mathbb{F}=\left\{\mathcal{F}_1,...,\mathcal{F}_n\right\}$ where $\mathcal{F}_i$ is a fail-prone system denoting the assumptions of process $p_i$.

Similarly, an asymmetric quorum system is a family of quorum systems $\mathbb{Q}=\left\{\mathcal{Q}_1,...,\mathcal{Q}_n\right\}$ where $\mathcal{Q}_i$ is the quorum system of process $p_i$. $\mathbb{Q}$ must be such that (a) for every two processes $p_i$ and $p_j$ and for every two quorums $Q_i$ of $p_i$ and $Q_j$ of $p_j$, $Q_i\cap Q_j$ contains a process which both $p_i$ and $p_j$ assume is not faulty, and (b) for every set of process $F_i$ that $p_i$ assume may fail together, there is a quorum of $p_i$ that is disjoint from $F_i$. Those conditions enable solving useful distributed computing tasks such as read-write registers and broadcast using well-known techniques. Moreover, Cachin and Tackmann show that property $Q^3(\mathcal{F})$ generalizes as follows TODO
%and that an asymmetric quorum system $\mathcal{F}$ that satisfies $Q^3(\mathcal{F})$ determines a canonical asymmetric quorum system... TODO

We could say that asymmetric quorum systems improve decentralization because there is no need to agree a-priori on the notion of quorum. However, satisfying property $Q^3(\mathcal{F})$ still requires prior coordination. Hence the question: how can a set of processes that initially have no notion of quorum system synchronize to obtain an asymmetric quorum system? Moreover, how can they do so in a decentralized manner, without using a central authority?

The Stellar Network offers a practical solution to this problem. However, as Cachin and Tackmann observe, the asymmetric quorum-system model does not capture Stellar's Federated Byzantine Quorum System model (the FBQS model for short). In this paper, we show that, similarly to symmetric and asymmetric quorum systems, a FBQS can be understood in terms of a fail-prone system.

%indeed, how are two processes that never talked to each other supposed to make failure assumptions that satisfy $Q^3(\mathcal{F})$?

% In a FBQS, processes do not know who the system participants are exactly. Instead, each process $p_i$ only knows of a subset $P_i$ of the participants and makes assumptions about failures using a fail-prone system $\mathcal{F}_i$ over $P_i$. Moreover, the meaning of the fail-prone system is different from previous work: process $p_i$ assumes that, in every execution, (a) it has made accurate assumptions, i.e. the set of process that actually fail is a set $F$ such that $P_i\cap F \in\mathcal{F}_i$, (b) the processes in $P_i\setminus F$ in turn have made accurate assumptions, and (c) so on recursively.

\section{Federated Quorum Systems}

In a FBQS, processes do not know who the system participants are exactly. Instead, each process $p_i$ only knows of a subset $P_i$ of the participants and makes assumptions about failures using a fail-prone system $\mathcal{F}_i$ over $P_i$. The processes' assumptions are satisfied in an execution if and only if, for every $p_j\in I$, the set of processes that actually fail is a set $F$ such that $P_j\cap F \in\mathcal{F}_j^*$\todo{the star denotes set of subsets}. To sum up, a federated fail-prone system as a family of fail-prone systems $\mathbb{F}=\left\{\left(P_1,\mathcal{F}_1\right),...,\left(P_n,\mathcal{F}_n\right)\right\}$ where $P_i$ is the set of processes that $p_i$ knows of and $\mathcal{F}_i$ is a fail-prone system over $P_i$ that denotes the assumptions that $p_i$ makes about the possible failures of members of $P_i$.

Note that two processes $p_i$ and $p_j$ may not even know of a common third process, i.e. it is well possible that $P_i\cap P_j=\emptyset$. As should become apparent in definition of Property $Q^3(\mathbb{F})$ below, to stay in agreement despite this apparent problem, processes in a FBQS rely on a transitive reachability property that must hold despite failures.

Given a federated fail-prone system $\mathbb{F}=\left\{\left(P_1,\mathcal{F}_1\right),...,\left(P_n,\mathcal{F}_n\right)\right\}$, define the directed graph $G_\mathbb{F}$ over processes such that there is an edge from $p_i$ to $p_j$ if and only if $p_j\in P_i$.\todo{if we need transitive closure, find another notation for set of subsets}% Let $G_\mathbb{F}^*$ be the transitive closure of $G_\mathbb{F}$, i.e. $G_\mathbb{F}^*(p_i)$ is the set of processes reachable from $p_i$ in $G_\mathbb{F}$.

To define a federated quorum systems for a fail-prone system $\mathbb{F}=\left\{\left(P_1,\mathcal{F}_1\right),...,\left(P_n,\mathcal{F}_n\right)\right\}$, we first need the notion of fail-prone set over a subset $S$ of the processes. Given a set of processes $S$, a fail-prone set over $S$ is a set of processes obtained by taking the union over all processes in $S$ of one fail-prone set per process. Formally, $T$ is a fail-prone set over $S$ if and only if there exists a family of fail-prone sets $\left\{F_1,...,F_n\right\}$ such that (a), for every $p_i\in S$, $F_i\in \mathcal{F}_i$ and (b) $T=\bigcup_{p_i\in S} F_i$.

\begin{definition}[Property $Q^3(\mathbb{F})$]\todo{this is not strong enough}
  Property $Q^3(\mathbb{F})$ holds if and only if for every two processes $p_i$ and $p_j$ and for every fail-prone set $F$ over $G_\mathbb{F}^*(p_i)\cup G_\mathbb{F}^*(p_j)$, there exists a path $x_i$ in $G_\mathbb{F}$ starting at $p_i$ and a path $x_j$ in $G_\mathbb{F}$ starting at $p_j$ such that
  \begin{enumerate}
    \item the paths $x_i$ and $x_j$ intersect, and
    \item $x_i$ and $x_j$ do not intersect $F$.
  \end{enumerate}
\end{definition}

We are now ready to define federated quorum systems for $\mathbb{F}$ and show that there exists a federated quorum system for $\mathbb{F}$ if and only if $Q^3(\mathbb{F})$ holds.


%As should become apparent in the Consistency property of federated quorum systems state below, to stay in agreement despite this apparent problem, processes in a FBQS rely on transitive reachability across the network.


A federated quorum system for $\mathbb{F}$ is a family of quorum systems $\mathbb{Q}=\left\{Q_1,...,Q_n\right\}$, where $Q_i$ is a quorum system over $\mathcal{P}$ for $p_i$. Moreover, $\mathbb{Q}$ must satisfy the following three properties.
\begin{itemize}
  \item [Closure] If $Q_i$ is a quorum of $p_i$ and $p_j\in Q_i$, then there exists a quorum $Q_j$ of $p_j$ such that $Q_j\subseteq Q_i$.\todo{needed?}
  % \item [Availability] If $Q$ is a quorum of $p_i$ then there exists a fault-prone set $F$ of $p_i$ such that $P_i\setminus F \subseteq Q$.
  \item [Availability] If $Q$ is a quorum of any process, then there exists a fault-prone set $F$ over $Q$ that is disjoint from $Q$.
  \item[Consistency] If $Q_i$ is a quorum of $p_i$ and $Q_j$ is a quorum of $p_j$, if $F_i$ is a fail-prone set of over $Q_i$, if $F_j$ is a fail-prone set over $Q_j$, then there is a process $p_k\in Q_i\cap Q_j$ such that
    \begin{itemize}
      \item there is a path $x_i$ in the graph $G_\mathbb{F}$ from $p_i$ to $p_k$ such that $x_i$ contains only nodes in $Q_i\setminus (F_i \cap F_j)$, and
      \item there is a path $x_j$ in the graph $G_\mathbb{F}$ from $p_j$ to $p_k$ such that $x_j$ contains only nodes in $Q_j\setminus (F_i \cap F_j)$.
    \end{itemize}
\end{itemize}

Intuitively, the Consistency property states that not only must the quorums of two processes $p_i$ and $p_j$ intersect despite failures, but their intersection must be reachable from $p_i$ and $p_j$ despite failures.

\begin{thm}
  There exists a federated quorum system for $\mathbb{F}$ if and only if $Q^3(\mathbb{F})$ holds.
\end{thm}

To prove the theorem, we first show that for every federated fail-prone system $\mathcal{F}$ satisfying $Q^3(\mathcal{F})$, there exists a canonical quorum system $\overline{\mathbb{F}}$ for $\mathbb{F}$. Second, we show that if $\mathbb{Q}$ is a federated quorum system for $\mathcal{F}$, then $Q^3(\mathcal{F})$ must hold.

\begin{definition}[Canonical federeated quorum system for $\mathbb{F}$]
Given a federated fail-prone system $\mathbb{F}$ satisfying $Q^3(\mathbb{F})$, we define $\overline{\mathbb{F}}=\left\{\mathcal{Q}_1,...,\mathcal{Q}_n\right\}$, where the set $\mathcal{Q}_i$ of quorums of $p_i$ is the biggest set such that, for every $Q\in\mathcal{Q}_i$,
\begin{enumerate}
  \item $p_i\in Q$ and
  \item for every $p_j\in Q$, there is a fault-prone set $F$ of $p_j$ such that $P_j\setminus F\subseteq Q$.
\end{enumerate}
\end{definition}


Clearly, $\overline{\mathbb{F}}$  satisfies the Availability property by definition. Moreover, observe that if $Q$ is a quorum of $p_i$ and $p_j\in Q$, then $Q$ is also a quorum of $p_j$; thus $\overline{\mathbb{F}}$ satisfies the Closure property. Finally, consider two processes $p_i$ and $p_j$ and two quorums $Q_i$ of $p_i$ and $Q_j$ of $p_j$. Then,

\section{Old stuff}


Thus, the meaning of the fail-prone system is different from previous work: process $p_i$ assumes that there is a set $I$ of processes, unknown to $p_i$, including $p_i$, such that, in every execution, every member of $I$ has made accurate assumptions; in other words, for every execution and every $p_j\in I$, the set of processes that actually fail is a set $F$ such that $P_j\cap F \in \mathcal{F}_j^*$.

As in the work of Cachin and Tackmann, a quorum system is... Now the canonical quorum system for $\mathcal{F}$ is as in Stellar, where the $F_i$s determine slices!

Formally, we define a permissionless asymmetric fail-prone system $\mathbb{F}=\left\{\left(P_1,\mathcal{F}_1\right),...,\left(P_n,\mathcal{F}_n\right)\right\}$ where $P_i$ is a set of processes and $\mathcal{F}_i$ is a set of sets of processes such that every $F\in\mathcal{F}_i$ is a subset of $P_i$.

A quorum system for $\mathbb{F}$ is a set of sets $\mathbb{Q}=\left\{Q_1,...,Q_m\right\}$, where $Q_i$ is a quorum system for $p_i$. $\mathbb{Q}$ must satisfy the following properties. The intersection of two quorums

  We now generalize the $B^3$ condition, abbreviated $B^3(\mathbb{F})$, as follows. First, define the directed graph $G_\mathbb{F}$ over processes such that there is an edge from $p_i$ to $p_j$ if and only if $p_j\in P_i$. Let $G_\mathbb{F}^*$ be the transitive closure of $G_\mathbb{F}$, and let $G_\mathbb{F}^*(p_i)$ be the set of processes reachable from $p_i$ in $G_\mathbb{F}$. Let $\mathcal{C}$ be the set of functions $\left\{C_1,...,C_m\right\}$ that map a process to one of its fail-prone sets (i.e. each $C_j$ is a function from processes to sets of processes such that, for every $p_i$, $C_j(p_i)\in \mathcal{F}_i$).
  $B^3(\mathbb{F})$ holds if and only if for every two processes $p_i$ and $p_j$, for every $C_\alpha$, $C_\beta$, $C_\gamma$, and $C_\delta$ in $\mathcal{C}$:
  \begin{equation}
    \mathcal{P}\not\subseteq \left(\bigcup_{x \in G_\mathbb{F}^*(p_i)}C_\alpha(x)\right) \cup \left(\bigcup_{y \in G_\mathbb{F}^*(p_j)}C_\beta(y)\right) \cup \left(\left(\bigcup_{a \in G_\mathbb{F}^*(p_i)}C_\gamma(a)\right) \cap \left(\bigcup_{b \in G_\mathbb{F}^*(p_j)}C_\delta(b)\right) \right)
  \end{equation}


The result of Cachin and Tackmann is of a similar form: a quorum system exists if, for every two processes, three sets do not cover the entire system. However, in a permissionless system,


The Stellar Network is a permissionless network where node's self-declared trust assumptions, embodied in their quorum slices, give rise to a set of quorums.

We consider a distributed system consisting of a large number of processes including $n$ well-behaved processes $\mathcal{P}=\left\{p_1,...,p_n\right\}$. We assume that processes do not know $\mathcal{P}$.

Informally, we consider the situation where individuals make decision, such as what to vote for, by listening to their friends.

may join what we call the Network in order to participate in a consensus protocol that will keep them synchronized with the Network. However, because the system is permissionless, processes do not know precisely which other processes belong to the Network. Nevertheless, to synchronize with the Network, a process $p_i$ selects a set of other processes $P_i$ that it believes to be synchronized with the Network, with the goal to follow what processes in $P_i$ indicate about the state of the Network. Since processes in $P_i$ may fail, $p_i$ also makes assumptions, in the form of a fail-prone system $\mathcal{F}_i$ over $P_i$, about the subset of $P_i$ that may fail together in an execution.

%This is a situation analogous to the Internet, where processes join the Internet through peering agreements, expecting to be able to become connected to the entire Internet, even though a node doesn't know exactly what the Internet precisely consist of.

Since node do not know exactly what the Network consists of, they cannot make sensible assumptions about which processes may fail together. Instead, processes make assumptions that are restricted to a few other nodes that they know of and that they expect to themselves make sensible assumptions.

Formally, we define a permissionless asymmetric fail-prone system $\mathbb{F}=\left\{\left(P_1,\mathcal{F}_1\right),...,\left(P_n,\mathcal{F}_n\right)\right\}$ where $P_i$ is a set of processes and $\mathcal{F}_i$ is a set of sets of processes such that every $F\in\mathcal{F}_i$ is a subset of $P_i$. $P_i$ is the set of nodes that $p_i$ expects to be sufficiently connected to the rest of the network, while $\mathbb{F}_i$ is a fail-prone system over $P_i$.

  We now generalize the $B^3$ condition, abbreviated $B^3(\mathbb{F})$, as follows. First, define the directed graph $G_\mathbb{F}$ over processes such that there is an edge from $p_i$ to $p_j$ if and only if $p_j\in P_i$. Let $G_\mathbb{F}^*$ be the transitive closure of $G_\mathbb{F}$, and let $G_\mathbb{F}^*(p_i)$ be the set of processes reachable from $p_i$ in $G_\mathbb{F}$. Let $\mathcal{C}$ be the set of functions $\left\{C_1,...,C_m\right\}$ that map a process to one of its fail-prone sets (i.e. each $C_j$ is a function from processes to sets of processes such that, for every $p_i$, $C_j(p_i)\in \mathcal{F}_i$).
  $B^3(\mathbb{F})$ holds if and only if for every two processes $p_i$ and $p_j$, for every $C_\alpha$, $C_\beta$, $C_\gamma$, and $C_\delta$ in $\mathcal{C}$:
  \begin{equation}
    \mathcal{P}\not\subseteq \left(\bigcup_{x \in G_\mathbb{F}^*(p_i)}C_\alpha(x)\right) \cup \left(\bigcup_{y \in G_\mathbb{F}^*(p_j)}C_\beta(y)\right) \cup \left(\left(\bigcup_{a \in G_\mathbb{F}^*(p_i)}C_\gamma(a)\right) \cap \left(\bigcup_{b \in G_\mathbb{F}^*(p_j)}C_\delta(b)\right) \right)
  \end{equation}

  as a family of
fail-prone systems, where the fail-prone system $\mathcal{F}_i$ is a set of
sets of processes that denote the trust assumptions of process $p_i$. However,
we interpret the assumption encoded by a fail-prone system differently from
Cachin and Tackmann.


Each fail-prone system $\mathcal{F}_i$ is a set of sets of
processors denoting process $p_i$'s assumption about.

However,

A node wants to join the Internet, but it doesn't know precisely what the
Internet is. Thus it makes assumption about who is likely to be connected in
the Internet. This assumption takes the form of a set $T$ of nodes that are
trusted to be well-connected, and a set of sets $\mathcal{F}\subseteq 2^{T}$ of
fail-prone subsets of $T$. From this we can derive a set of ``quorums'', which
we here call slices, in the usual way. What about the usual condition on the
$F$s? What we'd like to obtain is that if we take a set of nodes whose
assumptions are satisfied, then they're intact.

Seems like the notion of minimal quorum should show up somewhere.

What about the traditional theorem saying there exists a quorum system iff
there exists suitable fail-prone sets? Well, we need to define a quorum system
first. Or maybe it's a ``slice system''?

Let's say a slice system is a FBQS with quorum intersection and availability.
What's the theorem now?

Traditional quorum systems seem incompatible with the permissionless model. Even
heterogeneous, asymmetric quorum systems like the one of Cachin seem difficult
to deploy in a permissionless system because it is assumed that all quorums
intersect. How is this to be achieved without prior synchronization?

In contrast, The Federated Byzantine Quorum System (FBQS) model

Quorum system do not seem adapted to permissionless systems. For example,
threshold quorum systems define quorums as set of cardinality at least... This
refers to the total number of nodes. Even more general, asymmetric quorum
system make the assumption that quorums intersect.
What about the permissionless setting?

and a set of fail-prone sets.
Quorums and fail-prone sets are sets of nodes, and fail-prone sets restrict the
failures that can happen in a given execution to at most one fail-prone set.

A simple example is a set of $n$ nodes where every set of cardinality strictly
greater than $2n/3$ is a quorum and any set of cardinality strictly smaller
than $n/3$ is a fail-prone set. It is well-known that consensus is solvable in
such a system if we assume partial synchrony. The reason is that, in every
possible execution, every two quorums intersect in a well-behaved node, and
there is at least one fully well-behaved quorum.

Quorum systems seem ill-suited to permissionless systems because the notion of
quorum is a global notion that must be agreed upon by all nodes. Instead, what
if we allow each node to make its own mind about what a quorum is?


Obviously, nodes may choose non-intersecting quorum, in which case it will be
impossible to guarantee system-wide agreement. However, some subset of nodes
may choose suitable quorums, and we can expect the system to consist of
consensus islands immersed in a sea of diverging nodes.

Given the quorums chosen by each node and a set of fail-prone sets, what can we
say the ability of the nodes to solve consensus?

% \section{The Union Theorem}

% Is there a counter-example if the system does not satisfy the PBQS property?
% Consider 5 nodes numbered from 1 to 5.

% \begin{center}
% \begin{tabular}{ |c|c| }
  % \hline
  % node & quorums \\
  % \hline
  % $n_1$ & \{  \\
  % \hline
  % $n_2$ & cell5 \\
  % \hline
  % $n_3$ & cell8 \\
  % \hline
  % $n_4$ & cell8 \\
  % \hline
  % $n_5$ & cell8 \\
  % \hline
% \end{tabular}
% \end{center}

\end{document}
